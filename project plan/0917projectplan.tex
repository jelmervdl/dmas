\documentclass[a4paper]{article}

\usepackage[english]{babel}

% References
\usepackage[backend=bibtex]{biblatex}
\usepackage{csquotes}
\bibliography{biblio}

\title{Project Plan\\ \sc{traffic simulation self-driving cars}}
\author{%
	Laura van de Braak (s2165341) \and 
	Jelmer van der Linde (s1772791) \and 
	Laura Baakman (s1869140)
}

\begin{document}

\maketitle

\paragraph{Problem} How do self-driving cars influence traffic?

\paragraph{State of the art} \citeauthor{paruchuri2002multi} describe an agent simulation of organised traffic. They name a number of factors that play a part in how a driver reacts to traffic. Human drivers are imperfect, and can be distracted by external or internal factors. Alternatively, \citeauthor{jiang2010microscopic} created a simulation of autonomous driving, in their paper they describe how one can model autonomous vehicles.

\paragraph{New idea} We model several traffic situations where both self-driving cars and `normal' cars are part of the traffic. We will start out with a simple intersection where the normal rules apply. As we are further along in the project we will add more complex situations such as traffic lights and stop signs. Furthermore we might also look into different acceleration speeds for the `normal' cars and different drivers such as those proposed by \citeauthor{paruchuri2002multi}, or cars with (adaptive) cruise control.

We will base or model of autonomous cars on \citeauthor{jiang2010microscopic}, our humand drivers will be based on \citeauthor{paruchuri2002multi}.

\paragraph{Results} We expect that a high ratio of self-driving cars will result in fewer traffic jams, but that adding some self-driving cars to normal traffic will not change the flow of traffic significantly.  

\paragraph{Relevance} Self-driving cars are becoming more and more prevalent. Several companies are already testing their self-driving cars on the Californian streets \cite{honda}. These cars will change the flow of traffic. 

\printbibliography

\end{document}