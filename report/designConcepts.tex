%!TEX root = report.tex
% \algemeen{Bespreek (een subset van) de volgende elementen, niet noodzakelijk in deze volgorde.}
% Ik heb alles waar niet echt iets stond eruit gehaald om het stacato waar Bart het over had eruit te halen.

\vdbraak{Hier kan vrij snel ook uitleggen over gedragingen van de drivers en waarom, met iets aan theorie-achtergrond. Hiervoor moeten we eerst iets meer gedraging hebben.}

% \paragraph{Emergence} 
% % \algemeen{Which system-level phenomena truly emerge from individual traits, and which phenomena are merely imposed?}
% There are no system-level phenomena that emerge from individual traits.

% \paragraph{Adaptation} 
% % \algemeen{What adaptive traits do the model individuals have which directly or indirectly can improve their potential fitness, in response to changes in themselves or their environment?}
% The individuals have no adaptive traits.

% \paragraph{Fitness} 
% % \algemeen{Is fitness-seeking modelled explicitly or implicitly? If explicitly, how do individuals calculate fitness?}
% Fitness-seeking is not modelled explicitly. \todo[inline]{Ook niet echt impliciet toch?}

% \paragraph{Prediction} 
% % \algemeen{In estimating future consequences of their decisions, how do individuals predict the future conditions they will experience?}
% Agents are purely reactive, they do not not consider future conditions when deciding on their next action.  

\paragraph{Sensing \& Interaction} 
% \algemeen{What internal and environmental state variables are individuals assumed to sense or “know” and consider in their adaptive decisions?}
Individual vehicles can sense other vehicles and other objects if they are close enough to `see' them. They cannot see into each other's state variables, or those of the simulation. What is actual perceived by an agent depends on its type, self-driving versus human driver, and in the case of the human agent its state. 

Although vehicles can observe other vehicles and objects they cannot interact with them. 
\vdbraak{Dus wat kan een mens waarnemen?} 
\vdbraak{Wat kan een self-driving car waarnemen?}

% \paragraph{Interaction} 
% % \algemeen{What kinds of interactions among individuals are assumed?}
% Vehicles can observer other vehicles, but they cannot interact.

% \paragraph{Stochasticity} 
% % \algemeen{Is stochasticicity part of the model? What are the reasons?}
% In this simulation everything is determined, there is no stochasticity.

% \paragraph{Collectives} 
% % \algemeen{Are individuals grouped into some kind of collective, e.g. a social group?}
% There are no collectives in this simulation, all the agents function for themselves.

\paragraph{Observation} 
% \algemeen{How are data collected from the IBM for testing, understanding, and analysing it?}
The data we collect is with regard to the delay encountered while driving from one point to another, and the number of crashes. These measurements are discussed extensively in \cref{sub:method:design}.


