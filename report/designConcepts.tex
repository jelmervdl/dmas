\algemeen{Bespreek (een subset van) de volgende elementen, niet noodzakelijk in deze volgorde.}

\paragraph{Emergence} \algemeen{Which system-level phenomena truly emerge from individual traits, and which phenomena are merely imposed?}

\paragraph{Adaption} \algemeen{What adaptive traits do the model individuals have which directly or indirectly can improve their potential fitness, in response to changes in themselves or their environment?}

\paragraph{Fitness} \algemeen{Is fitness-seeking modelled explicitly or implicitly? If explicitly, how do individuals calculate fitnes?}

\paragraph{Prediction} \algemeen{In estimating future consequences of their deci- sions, how do individuals predict the future conditions they will experience?}

\paragraph{Sensing} \algemeen{What internal and environmental state variables are individualsassumedto sense or “know”and consider in their adaptive decisions?}

\paragraph{Interaction} \algemeen{What kinds of interactions among individuals are assumed?}

\paragraph{Stochasticity} \algemeen{Is stochasticicity part of the model? What are the reasons?}

\paragraph{Collectives} \algemeen{Are individuals grouped into some kind of collec- tive, e.g. a social group? Observation:}

\paragraph{Observation} \algemeen{How are data collected from the IBM for testing, understanding, and analyzing it?}
