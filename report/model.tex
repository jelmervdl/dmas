%!TEX root = report.tex

This model description follows the ODD format described in \textcite{grimm2006standard}. We start with a general overview of the agent-based model, followed by explanations of the design concepts, after which we will discuss the details of the model. 

\subsubsection{Overview}
\label{subsub:method:model:overview}

	\paragraph{Purpose}
	\label{par:method:model:overview:purpose}
	% \todo[inline]{A clear, concise and specific formulation of the model’s purpose. This element informs about why you need to build a complex model, and what, in general and in particular, you are going to do with your model.}

	The purpose of this model is to have (build?) a traffic simulation in which we can test whether autonomous vehicles joining regular `human' traffic has an influence on the traffic delay experienced. To do this, we need a simulation in which we have different types of agents, the autonomous and the `human' ones, and a space that emulates a traffic simulation. In that simulation, we will let both types of agents drive over the roads, encounter intersections, and we will measure the delay an number of crashes. This we will do with different ratios of autonomous to human vehicles. 

	\paragraph{State variables and scales}
	\label{par:method:model:overview:state}

	This model contains two different hierarchical levels, the Car/Driver level and the traffic level. The traffic level contains the possible roads and paths, and all the vehicles. The driver level contains all parameters necessary for the functioning of a vehicle.  

	% The dimensions must be clearly defined for all parameters and variables in the tables.
		\subparagraph{Low-level state variables}
		The parameters described in table \ref{tab:par:method:model:overview:state:lowlevel:car} are those contained by all drivers. 

		\todo[inline]{low-level variables that characterize the low-level entities of the model, i.e. individuals, table? zie parameters Car en Driver}

		\todo[inline, color=yellow]{Check that the parameters are still accurate}

		\todo[inline, color=green]{Turn into book tabs table}

		\begin{table}[H]
			\centering
			\begin{tabular}{|l | p{10.5cm} |}
				\hline
				Parameter				& Purpose \\ \hline \hline
				acceleration 			& Indicates if a vehicle has forward, backward, or no acceleration. \\ \hline
				targetBodyAngle 		& The angle at which the target is located, relative to the body of the vehicle. \\ \hline
				maxSteerAngleDeg 		& The maximum angle the wheels can turn. \\ \hline
				power 					& The impulse power of a vehicle. \\ \hline
				wheelAngleDeg 			& The current angle of the wheel, relative to the body of the vehicle. \\ \hline
				steeringSpeed 			& The speed with which a vehicle can turn (the turning radius). \\ \hline
				visionRange 			& The range of the vision of the vehicle. \\ \hline
				width 					& The width of the vehicle. \\ \hline
				length 					& The length of a vehicle. \\ \hline
				% colour 				& The colour of a vehicle \\ \hline
				driver 					& The driver agent. \\ \hline
				initialPosition 		& The initial position of the agent at the start of the simulation. \\ \hline
				% bodyFixture 			& The `fixture' of the bocy of a vehicle, used when viewing other vehicles\\ \hline
				% visionFixture 		& The `fixture of the vision of the vehicle, describing what a vehicle can see'\\ \hline
				targetSpeedKMH			& The maximum speed of a vehicle. \\ \hline
				path					& The path the agent is going to follow. \\ \hline
			\end{tabular}
			\caption{Car-owned state parameters}
			\label{tab:par:method:model:overview:state:lowlevel:car}
		\end{table}

		\subparagraph{Higher-level entities}
		\todo[inline]{Aantal autos's en meer van dat soort blaat, dingen die hele model beïnvloed ipv een agent.}

		\begin{table}[H]
			\centering
			\begin{tabular}{|l | p{10.5cm} |}
				\hline
				Parameter				& Purpose \\ \hline \hline
				cars 					& A list of the cars in the simulation. \\ \hline
				streetGraph		 		& A graph representation of the network of streets. \\ \hline
			\end{tabular}
			\caption{simulation-owned state parameters}
			\label{tab:par:method:model:overview:state:highlevel:sim}
		\end{table}


		\subparagraph{Scales}
		\todo[inline]{Length of time steps and time horizon. The reason why these scales have been selected should briefly be explained.}		

	\paragraph{Process overview and scheduling}
	\label{par:method:model:overview:process}
	\todo[inline]{Welke processen zitten in het  model, eventueel met tabel. }
	\todo[inline]{bespreek scheduling van processen, welke volgorde uitgevoerd?}
	\todo[inline]{How is time modeled}
	\todo[inline]{How are actions that happen concurrently IRL executed in the model}
	\todo[inline]{Are some actions executed in a random order?}


\subsubsection{Design Concepts}
\label{subsub:method:model:design}
	\todo[inline]{Bespreek (een subset van) de volgende elementen, niet noodzakelijk in deze volgorde.}

	\paragraph{Emergence} \todo[inline]{Which system-level phenomena truly emerge from individual traits, and which phenomena are merely imposed?}

	\paragraph{Adaption} \todo[inline]{What adaptive traits do the model individuals have which directly or indirectly can improve their potential fitness, in response to changes in themselves or their environment?}

	\paragraph{Fitness} \todo[inline]{Is fitness-seeking modelled explicitly or implicitly? If explicitly, how do individuals calculate fitnes?}

	\paragraph{Prediction} \todo[inline]{In estimating future consequences of their deci- sions, how do individuals predict the future conditions they will experience?}

	\paragraph{Sensing} \todo[inline]{What internal and environmental state variables are individualsassumedto sense or “know”and consider in their adaptive decisions?}

	\paragraph{Interaction} \todo[inline]{What kinds of interactions among individuals are assumed?}

	\paragraph{Stochasticity} \todo[inline]{Is stochasticicity part of the model? What are the reasons?}

	\paragraph{Collectives} \todo[inline]{Are individuals grouped into some kind of collec- tive, e.g. a social group? Observation:}

	\paragraph{Observation} \todo[inline]{How are data collected from the IBM for testing, understanding, and analyzing it?}

\subsubsection{Details}
\label{subsub:method:model:details}

	\paragraph{Initialization}
	\label{par:method:model:details:initialization}
	\todo[inline]{How are the environment and the individuals created at the start of a simulation run, i.e. what are the initial values of the state variables? Is initial- ization always the same, orwas it varied among simulations? Were the initial values chosen arbitrarily or based on data? References to those data should be provided}

	\paragraph{Input}
	\label{par:method:model:details:input}
	\todo[inline]{The dynamics of many IBMs are driven by some environmental conditions which change over space and time. All these environmental conditions are “input”, i.e. imposed dynamics of certain state variables.}

	% \paragraph{Submodels}
	% \label{par:method:model:details:submodels}
	%  No clue
