%!TEX root = report.tex
% \algemeen{Bespreek (een subset van) de volgende elementen, niet noodzakelijk in deze volgorde.}
% Ik heb alles waar niet echt iets stond eruit gehaald om het stacato waar Bart het over had eruit te halen.


\paragraph{Agents}
There are a couple differences between human drivers and autonomous agents. Firstly, the view of the human drivers is limited, compared to that of autonomous drivers. Where the autonomous drivers have a circular view in a radius of 12 meters, the human drivers have an arc-shaped view of differing radius. This radius differs per driver. As the drivers are built, they get assigned a \t{viewLength}. This is sampled from a random distribution with a mean of 8 meters and a standard deviation of 2. If the sampled values are below zero, they are clamped to zero, to avoid negative a negative \t{viewLength}. 

Furthermore, human drivers have an \t{actPeriod} of \SI{100}{\milli\second}, which means that \SI{100}{\milli\second} pass between each update step. This causes them to have a slower reaction to the environment than autonomous drivers, which have an \t{actPeriod} of \SI{0}{\milli\second}, which means they always react instantly. 

Lastly, the way drivers react to other vehicles in their neighbourhood differs, this process is discussed in  \cref{par:method:model:overview:process}. 

% \paragraph{Emergence} 
% % \algemeen{Which system-level phenomena truly emerge from individual traits, and which phenomena are merely imposed?}
% There are no system-level phenomena that emerge from individual traits.

% \paragraph{Adaptation} 
% % \algemeen{What adaptive traits do the model individuals have which directly or indirectly can improve their potential fitness, in response to changes in themselves or their environment?}
% The individuals have no adaptive traits.

% \paragraph{Fitness} 
% % \algemeen{Is fitness-seeking modelled explicitly or implicitly? If explicitly, how do individuals calculate fitness?}
% Fitness-seeking is not modelled explicitly. \todo[inline]{Ook niet echt impliciet toch?}

% \paragraph{Prediction} 
% % \algemeen{In estimating future consequences of their decisions, how do individuals predict the future conditions they will experience?}
% Agents are purely reactive, they do not not consider future conditions when deciding on their next action.  

\paragraph{Sensing \& Interaction} 
% \algemeen{What internal and environmental state variables are individuals assumed to sense or “know” and consider in their adaptive decisions?}
Individual vehicles can sense other vehicles and other objects if they are close enough to `see' them. Whether or not a driver is close enough to observe an object depends on its \t{viewLength}. They cannot see into each other's state variables, or those of the simulation. What is actually perceived by an agent depends on its type, self-driving versus human driver, and in the case of the human agent its state. 
\vdbraak{Hoe hangt het van hun state af?}

Although drivers can observe other vehicles and objects they cannot interact with them. Both humans and autonomous vehicles can only see other objects if they appear within that agents' field of view. The field of view of an autonomous driver is larger, and is updated more often, consequently they observe more.

% \paragraph{Interaction} 
% % \algemeen{What kinds of interactions among individuals are assumed?}
% Vehicles can observer other vehicles, but they cannot interact.

% \paragraph{Stochasticity} 
% % \algemeen{Is stochasticicity part of the model? What are the reasons?}
% In this simulation everything is determined, there is no stochasticity.

% \paragraph{Collectives} 
% % \algemeen{Are individuals grouped into some kind of collective, e.g. a social group?}
% There are no collectives in this simulation, all the agents function for themselves.

\paragraph{Observation} 
% \algemeen{How are data collected from the IBM for testing, understanding, and analysing it?}
We measure how long a vehicle takes to get from its point of departure to its destination. The exact measurements  are discussed extensively in \cref{sub:method:design}.


