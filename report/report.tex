\documentclass[draft, twoside]{article}
\usepackage{./style/bnaic}

\usepackage[english]{babel}

% Floats
\usepackage{float}

%Images
\usepackage{tikz}
\usepackage{graphicx}
\usepackage[hang, small,labelfont=bf,up,textfont=sl,up, justification=raggedright]{caption}
\usepackage{subcaption}
\DeclareCaptionLabelFormat{opening}{(#2)}
\captionsetup{subrefformat=opening}

% Units
\usepackage{siunitx}

% Tables
\usepackage{booktabs}

% For ttfamily in Tables
\usepackage{array}

% References within paper
\usepackage{varioref}
\usepackage{hyperref}
\usepackage[noabbrev]{cleveref}


% References to other papers
\usepackage[backend=bibtex, citestyle=authoryear]{biblatex}
\usepackage{csquotes}
\bibliography{biblio}

% Temp
\usepackage[obeyDraft, colorinlistoftodos, figwidth=\textwidth]{todonotes}
% \usepackage{showlabels}

% Other commands
\renewcommand{\t}[1]{\texttt{#1}}

% TODO commands
\newcommand{\jelmer}[1]{\todo[inline, color=red]{\textbf{Jelmer:} #1}}
\newcommand{\later}[1]{\todo[inline, color=yellow]{\textbf{Aan het eind:} #1}}
\newcommand{\algemeen}[1]{\todo[inline]{#1}}
\newcommand{\baakman}[1]{\todo[inline, color=green]{\textbf{Baakman:} #1}}
\newcommand{\vdbraak}[1]{\todo[inline, color=cyan]{\textbf{Van de Braak:} #1}}

\title{Simulating Autonomous and Human-Driven Vehicles in Traffic Intersections}
\author{
	Laura van de Braak (s2165341)\and
    Jelmer van der Linde (s1772791)\and
    Laura Baakman (s1869140)}

\pagestyle{empty}

\begin{document}

\algemeen{Algemene taken}
\later{Kijken of dit aan het einde nog klopt ivm met groei en updaten programma}
\baakman{Laura Baakman}
\vdbraak{Laura van de Braak}
\jelmer{Jelmer van der Linde}

\listoftodos
\clearpage

\maketitle

%!TEX root = report.tex
\begin{abstract}
\noindent 
\todo[inline]{Write some abstract.}
\end{abstract}

\algemeen{Ik mis nog dingen over wat de auto's precies doen in het model, dus dat ze op punt x aangemaakt worden, dan een random destination toegwezen krijgen, en hoe het pad naar de destination gevonden wordt. Weet ook niet zo goed waar we dat zouden moeten plaatsen.}
\algemeen{Bart had het heel erg over een voorspelling van resultaten, die moet ook nog ergens erin, denk in de inleiding, ofzo.}

\section{Introduction}
\label{sec:introduction}
%!TEX root = report.tex
Drivers in California already have to share their roads with a few of the self-driving cars that are being tested there. It will not be long before autonomous cars are ubiquitous. One might wonder how these changes in traffic influence the flow of traffic. A lot of research has been done on how one might regulate traffic differently given that most or all of the cars are autonomous. However this research generally proposes changes in infrastructure, and then models the influence of changing the ratio of autonomous cars versus human drivers. This paper focusses on the question of how adding autonomous cars to `normal traffic' without changing the infrastructure influences the flow of traffic. Before discussing our model and our results we first introduce the current state of the art in \cref{sub:intro:state_of_the_art} and further expand the contribution of this paper in \cref{sub:intro:new_idea}.

\subsection{State of the Art}
\label{sub:intro:state_of_the_art}
\textcite{dresner2007sharing} propose a hybrid intersection control mechanism that allows for both human and autonomous drivers. This proposal is a follow-up of several earlier papers, primarily their earlier paper, \cite{dresner2005traffic}. In this paper they propose a new infrastructure for autonomous intersection management which uses a multi-agent reservation system, consisting of two types of agents: the intersection managers and the driver agents. The intersection managers each manage an intersection, and are responsible for directing the driver agents through their intersection. The driver agents control their vehicles and communicate with intersection managers to get through intersections safely. The intersection managers have an intersection control policy, which they use to determine if the space to pass through requested by a driver agent is free at that time. If so, they confirm the reservation, and if not, they decline. This system was shown to work well and decrease traffic delays. 

% new idea
However, they found that while system described above works well when all agents in the simulation are autonomous, it is not yet applicable to the real world, since currently hardly any vehicles are autonomous. To solve this problem they looked into different policies concerning combining autonomous vehicles with human driven vehicles. In this hybrid system human drivers used the traffic lights, and self-driving cars used the reservation system described above.

% results
\textcite{dresner2007sharing} showed that some of the policies also resulted in a decrease of delays in traffic. The largest decrease was found when the autonomous vehicle to human ratio was high, however the self-driving car in this scenario encountered significantly less delays than their human driven counterpart due their use of the reservation system.

% relevance
A future with self-driving cars is becoming more realistic. To fully benefit from the emergence of autonomous vehicles our traffic system should aim to use their high-precision abilities to create more efficient, fast, and safe auto transportation.


\subsection{New Idea}
\label{sub:intro:new_idea}	
\algemeen{Eerste zin is iets verbeterd, maar kan nog steeds beter of anders.}
Where \textcite{dresner2007sharing} alter every intersection and every vehicle, we propose to change nothing in the existing infrastructure and look at whether the presence of autonomous cars by themselves can make a difference in traffic delays and crashes. 

\later{Updaten als we meer fancy shit in ons model hebben.}
We model several traffic situations where both self-driving cars and `normal' cars are part of the traffic using a simple intersection, without any of the extra infrastructure proposed by earlier research. 

% We start with a simple intersection where normal traffic rules apply. As we are further along in the project we will add more complex situations such as traffic lights and stop signals. Furthermore we might also look into different acceleration speeds for the `normal' cars and different drivers such as those proposed by \textcite{paruchuri2002multi}, or cars with (adaptive) cruise control.

Our model of autonomous cars is loosely based \textcite{jiang2010microscopic}, our human drivers are based on \textcite{paruchuri2002multi}.

\algemeen{Moet hier dan de verwachtte resultaten?}

This text is organised as follows: the next section discusses the model and our experiment in detail. \Cref{sec:results} presents and discusses the results of our experiment and \cref{sec:conclusion} concludes the paper with the implications of our results.

\section{Method}
\label{sec:method}
%!TEX root = report.tex
\algemeen{Introductie voor method sectie}

\subsection{Simulation Model}
\label{sub:method:model}
%!TEX root = report.tex
\todo[inline]{We use the protocol ODD proposed by \textcite{grimm2006standard} to describe our simulation model.}

\subsubsection{Overview}
\label{subsub:method:model:overview}

	\paragraph{Purpose}
	\label{par:method:model:overview:purpose}
	\todo[inline]{A clear, concise and specific formulation of the model’s purpose. This element informs about why you need to build a complex model, and what, in general and in particular, you are going to do with your model.}

	\paragraph{State variables and scales}
	\label{par:method:model:overview:state}
	% The dimensions must be clearly defined for all parameters and variables in the tables.
		\subparagraph{Low-level state variables}
		\todo[inline]{low-level variables that characterize the low-level entities of the model, i.e. individuals, table?}

		\subparagraph{Higher-level entities}
		\todo[inline]{Aantal autos's en meer van dat soort blaat, dingen die hele model beïnvloed ipv een agent.}

		\subparagraph{Scales}
		\todo[inline]{Length of time steps and time horizon. The reason why these scales have been selected should briefly be explained.}		

	\paragraph{Process overview and scheduling}
	\label{par:method:model:overview:process}
	\todo[inline]{Welke processen zitten in het  model, eventueel met tabel. }
	\todo[inline]{bespreek scheduling van processen, welke volgorde uitgevoerd?}
	\todo[inline]{How is time modeled}
	\todo[inline]{How are actions that happen concurrently IRL executed in the model}
	\todo[inline]{Are some actions executed in a random order?}


\subsubsection{Design Concepts}
\label{subsub:method:model:design}
	\todo[inline]{Bespreek (een subset van) de volgende elementen, niet noodzakelijk in deze volgorde.}

	\paragraph{Emergence} \todo[inline]{Which system-level phenomena truly emerge from individual traits, and which phenomena are merely imposed?}

	\paragraph{Adaption} \todo[inline]{What adaptive traits do the model individuals have which directly or indirectly can improve their potential fitness, in response to changes in themselves or their environment?}

	\paragraph{Fitness} \todo[inline]{Is fitness-seeking modelled explicitly or implicitly? If explicitly, how do individuals calculate fitnes?}

	\paragraph{Prediction} \todo[inline]{In estimating future consequences of their deci- sions, how do individuals predict the future conditions they will experience?}

	\paragraph{Sensing} \todo[inline]{What internal and environmental state variables are individualsassumedto sense or “know”and consider in their adaptive decisions?}

	\paragraph{Interaction} \todo[inline]{What kinds of interactions among individuals are assumed?}

	\paragraph{Stochasticity} \todo[inline]{Is stochasticicity part of the model? What are the reasons?}

	\paragraph{Collectives} \todo[inline]{Are individuals grouped into some kind of collec- tive, e.g. a social group? Observation:}

	\paragraph{Observation} \todo[inline]{How are data collected from the IBM for testing, understanding, and analyzing it?}

\subsubsection{Details}
\label{subsub:method:model:details}

	\paragraph{Initialization}
	\label{par:method:model:details:initialization}
	\todo[inline]{How are the environment and the individuals created at the start of a simulation run, i.e. what are the initial values of the state variables? Is initial- ization always the same, orwas it varied among simulations? Were the initial values chosen arbitrarily or based on data? References to those data should be provided}

	\paragraph{Input}
	\label{par:method:model:details:input}
	\todo[inline]{The dynamics of many IBMs are driven by some environmental conditions which change over space and time. All these environmental conditions are “input”, i.e. imposed dynamics of certain state variables.}

	% \paragraph{Submodels}
	% \label{par:method:model:details:submodels}
	%  No clue



\subsection{Experiment Design}
\label{sub:method:design}
\algemeen{Wat meten we enzo}


\section{Results}
\label{sec:results}
%!TEX root = report.tex
\algemeen{Introductie voor results}

\subsection{Experimental Findings}
\label{sub:results:experimental}
\algemeen{Wat hebben en gemeten, plaatjes enzo}

\subsection{Interpretation of Findings}
\label{sub:results:interpretation}
\algemeen{Blaat over plaatjes, misschien samentrekken met \cref{sub:results:experimental}.}


\section{Conclusion}
\label{sec:conclusion}
%!TEX root = report.tex

\jelmer{Misschien nog een koppelende zin a het vorige stuk? Ik mis nu het vorige stuk om aan te koppelen, dus dat wordt wat lastig.}

This section provides a discussion on the interpretation of the findings, as reported in \Cref{sub:results:interpretation}. \Cref{sub:conclusion:discussion} contains that discussion, and the relevance of these findings is discussed in \Cref{sub:conclusion:relevance}. 

\subsection{Discussion}
\label{sub:conclusion:discussion}
\jelmer{Discussie, conflicteert wat wij gevonden hebben met de resultaten van anderen?}




This simulation is not a realistic rendition of the real world. This is mostly because the simulation is on too small a scale, with a too low complexity.
\jelmer{Verder toelichten hoe complexity en scale te simpel zijn en wat gedaan kan worden om die realistischer te maken}
 Also, because more information concerning differences between human and autonomous drivers could be used in the simulation. 
\jelmer{Verder toelichten hoe we beter human en autonomous drivers kunnen onderscheiden.} 

To make it more realistic, more complex behaviours would need to be added. Furthermore, we have need of a enlarging of the complexity of the model.


\subsection{Relevance}
\label{sub:conclusion:relevance}

In this research we have showen that there will be a difference in delay from traffic as the ratio of autonomous to human cars increases. As autonomous cars are becoming more prevalent, we now have an idea about the changes that will occur in traffic situations in the years to come. 

\jelmer{Waarom is wat wij gevonden hebben relevant?, klopt wat Laura hierboven geschreven heeft met de echte resultaten?}


\printbibliography

\end{document}