%!TEX root = report.tex
Drivers in California already share their roads with the few self-driving cars that are tested there. It will not be long before autonomous cars are ubiquitous. One might wonder how adding self-driving cars changes the flow of traffic. A lot of research has been done on how traffic regulation could be improved given that most or all of the cars are autonomous. However this research generally proposes significant changes in infrastructure, and then models the influence of changing the ratio of autonomous cars versus human drivers. This paper focusses on the question of how adding autonomous cars to `normal traffic', without changing the infrastructure, influences the flow of traffic. Before discussing our model and results we first introduce the current state of the art in \cref{sub:intro:state_of_the_art} and further expand the contribution of this paper in \cref{sub:intro:new_idea}.

\subsection{State of the Art}
\label{sub:intro:state_of_the_art}
\textcite{dresner2007sharing} propose a hybrid intersection control mechanism that allows for both human and autonomous drivers. This proposal is a follow-up of several earlier papers, primarily of \textcite{dresner2005traffic}. In this paper they propose a new infrastructure for autonomous intersection management which uses a multi-agent reservation system, consisting of two types of agents: the intersection managers and the driver agents. The intersection managers each manage an intersection, and are responsible for directing the driver agents through their intersection. Each driver agent controls its vehicle and communicates with intersection managers to get through crossroads safely. The intersection managers have an intersection control policy, which they use to determine if the space required by a driver agent who wants to pass through is free at the requested time. If that is the case they confirm the reservation, and if not, they decline. This system was shown to work well and decrease traffic delays. 

% new idea
However, while the system described above works well, when all agents in the simulation are autonomous, it is not yet applicable to the real world, since currently hardly any vehicles are autonomous. To solve this problem they looked into different policies that combine autonomous vehicles with `normal' cars. In this hybrid system human drivers used the traffic lights, and self-driving cars used the reservation system described above.

% results
\textcite{dresner2007sharing} showed that some of the policies also resulted in a decrease of delays in traffic. The largest decrease was found when the autonomous vehicle to human ratio was high, however the self-driving car in this scenario encountered significantly less delays than their human driven counterpart due their use of the reservation system.

% relevance
A future with self-driving cars is becoming more realistic. To fully benefit from the emergence of autonomous vehicles our traffic system should aim to use their high-precision abilities to create more efficient, fast, and safe auto transportation.


\subsection{New Idea}
\label{sub:intro:new_idea}	
Where \textcite{dresner2007sharing} system alters every intersection and vehicle, we propose to change nothing in the existing infrastructure and look at whether the presence of autonomous cars by themselves can make a difference in the duration of traffic delays and the number of crashes. 

We model several traffic situations where both self-driving cars and `normal' cars are part of the traffic using a simple intersection, without any of the extra infrastructure proposed by earlier research. We expect to find a positive correlation between the ratio of human driven vehicles to autonomous cars and the duration of traffic delays.

This text is organised as follows: the next section discusses the model and our experiment in detail. \Cref{sec:results} presents and discusses the results of our experiment and \cref{sec:conclusion} concludes the paper with the implications of our results.