%!TEX root = report.tex

\jelmer{Misschien nog een koppelende zin a het vorige stuk? Ik mis nu het vorige stuk om aan te koppelen, dus dat wordt wat lastig.}

This section provides a discussion on the interpretation of the findings, as reported in \Cref{sub:results:interpretation}. \Cref{sub:conclusion:discussion} contains that discussion, and the relevance of these findings is discussed in \Cref{sub:conclusion:relevance}.

\subsection{Discussion}
\label{sub:conclusion:discussion}
We expected the flow of traffic to improve as more of the traffic flow will consist of autonomously driven vehicles. This expectation is based on that these kinds of vehicles are purpose-built for traffic conditions and have access to sensors and real-time processing power which are more suited than the general purpose senses of human drivers. On top of that autonomous vehicles have less variation in their driving properties and do not suffer from distractions in a similar way to humans.

Based on this assumption, we modelled the distinction between autonomous vehicles and human drivers on a difference in how much they can observe of the environment and how fast they can process that information. The other parameters, which consists mainly of the driving behaviours, are kept the same among both driver types because we expect that the driving behaviour of near-future autonomous vehicles will be modelled after the experiences of current human drivers.

The performed simulation did not show an improvement in the flow of traffic, and we can therefore not make any conclusions on whether our assumption, namely that the flow of traffic will improve with the increase partaking of autonomous vehicles. If anything, if our simulation of the distinction between the two types of traffic is in any way realistic, the results of this simulation can be interpreted as a sign that integration of autonomous traffic will not have any noticeable effect on the flow of traffic.

The simulation also did not show a clear distinction between the performance of autonomous vehicles versus human driven vehicles. This leads us to believe that the distinctive properties we have chosen to model, namely the different field of view and slower reaction times for human drivers, are not the distinctive properties that will define the characteristics of an autonomous vehicle. Only in the scenario where traffic zip-merged in to another lane we observed a difference. Here autonomous vehicles were at a disadvantage because the lack of information of the human drivers gave them an edge in assertiveness, taking advantage of the safe driving behaviour of the autonomous vehicles.

This simulation is not a realistic rendition of the real world. This is mostly because the simulation is on too small a scale, with a too low complexity. Cues human drivers use to navigate, such as road markings and signs, are left out. Nor is any complex driving behaviour implemented, like using expectations on what other drivers in a driver's environment are going to do and how that is going to influence the traffic situation. Since the usage of cues and of expectations are heavily emphasized by driving instructors, one can assume these abilities have a strong influence in the driving behaviour of drivers, and indirectly on the flow of traffic.

\subsection{Relevance}
\label{sub:conclusion:relevance}

In this research we have showen that there will be a difference in delay from traffic as the ratio of autonomous to human cars increases. As autonomous cars are becoming more prevalent, we now have an idea about the changes that will occur in traffic situations in the years to come. 

\jelmer{Waarom is wat wij gevonden hebben relevant?, klopt wat Laura hierboven geschreven heeft met de echte resultaten?}
