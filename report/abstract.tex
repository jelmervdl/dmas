%!TEX root = report.tex
\begin{abstract}
\noindent 
	% 1. What is the problem addressed?
	More and more autonomous vehicles are entering traffic. 
	% 2. What is the state of the art concerning this problem? 
	This has been an incentive to research how self-driving cars joining that join their human-steered counter parts effects the flow of traffic, most notably by \citeauthor{dresner2007sharing}. However, as far as we could find, all simulations in this area propose significant changes in infrastructure. Although changing infrastructure may make sense in the future, if autonomous cars are more commonplace, currently it is quite unlikely that policy makers will invest in changed infrastructure. 
	% 3. What is the new idea for addressing the problem?
	Our simulation aims to quantify some of the effects of the ratio human driven vehicles to autonomous cars on the flow of traffic. We model the differences between autonomous and human driven vehicles using different field of views and reaction times for both type of drivers.
	% 4. What are the results (expected or established)?
	We expect the flow of traffic to improve as the number of self-driving cars increases relative to the number of `normal' cars. However, our simulation shows barely any difference between both types of drivers, from which we conclude that these two properties field of view and reaction time are not the most distinctive properties.
	% 5. What is the relevance of this work?
	Even though we cannot answer our core question our research is a first step in finding the distinctive properties of self-driving cars which will become relevant to guide integration as soon as these types of vehicles join the normal traffic.
\end{abstract}