%!TEX root = report.tex 
\paragraph{Initialisation}
\label{par:method:model:details:initialization}
% \algemeen{How are the environment and the individuals created at the start of a simulation run, i.e. what are the initial values of the state variables? Is initialisation always the same, or was it varied among simulations? Were the initial values chosen arbitrarily or based on data? References to those data should be provided}
The initial variables are partly fixed, partly variable, an overview of all variables relating to cars and their initial values is presented in \cref{tab:par:method:model:details:init:car:value}. The variability is between agents, but not necessarily between simulations. These varying values are chosen arbitrarily for each run of the simulation.
\jelmer{Met java rand? Welke distirbutie samplet die? Als het een linear congruential generator is, is er een maximuum aantal random nummers voor hij herhaalt, hebben wij daar last van? Jelmer: Op het moment is niets willekeurig, behalve de keuze of er een Human driver of een Autonomous Driver wordt geïnstantieerd (middels Random.nextBoolean(), maar dat wordt anders), en wat de bestemming van de nieuwe bestuurder is. (middels Collections.shuffle()) Dit verhaal omzetten in verslag tekst.}


% Initial position
The \t{initialPosition} of a car, i.e. its point of departure, is uniformly and independently sampled from the \t{sources} of the \t{StreetGraph}. 

% Path
The \t{path} of a car starts at the \t{initialPosition} of the car and ends at its destination which is uniformly and independently sampled from the \t{sinks} of the \t{streetGraph}.

% Drivers
The type of \t{driver} is chosen randomly, weighted with the ratio of human driven vehicles to autonomous cars. I.e. if we have nine human drivers for each autonomous car, a driver is nine times as likely to get be driven by a human. 

% Car size
We sample the width and length of a car independently from a uniform distribution. The minimum and the maximum of this distribution are chosen in such a way that cars are no smaller than a 2004 Smart Coupe and no larger than the maximum size of a private car allowed by the Dutch legislation. The exact ranges are presented in \cref{tab:par:method:model:details:init:car:value}.


\begin{table}
	\centering
	\begin{tabularx}{\textwidth}{>{\ttfamily}lX}
		\toprule
		\normalfont{Parameter}	& Value \\  
		\midrule
		acceleration 			
			& \t{NONE} \\ 
		targetBodyAngle 		
			& \SI[mode=text]{0}{\radian} \\ 
		maxSteerAngleDeg 		
			& \SI[mode=text]{40}{\radian} \\ 
		power 					
			& \SI[mode=text]{30}{?}\\ 
		brakePower 		
			& \SI{50}{\meter\per\square\second}\\
		wheelAngleDeg 			
			& \SI[mode=text]{0}{\radian} \\ 
		steeringSpeed 			
			& \SI[mode=text]{5}{\meter\per\second} \\ 
		visionRange 			
			& \SI[mode=text]{8}{\meter} \\ 
		width 					
			& differs per vehicle, in the range \SIrange{1.470}{2.55}{\meter} \\ 
		length 					
			& differs per vehicle, in the range \SIrange{2.540}{12.0}{\meter} \\ 
		driver 					
			& a driver object, determining the agent type \\
		initialPosition 		
			& differs per vehicle. \\ 
		targetSpeedKMH			
			& \SI[mode=text]{0}{\kilo\meter\per\hour} \\ 
		path					
			& differs per vehicle \\ 		
		\bottomrule
	\end{tabularx}
	\caption{An overview of the initialisation of car-owned state parameters.}
	\label{tab:par:method:model:details:init:car:value}
\end{table} 

\jelmer{Hebben we geen vaste init parameters voor de simulatie? En als het random is, waaorm is het ranodm? Random uit welke distributie. Wat is constant, welke waarde heft het dan? Waarom die waarden?}

\jelmer{Wat zijn de simulation variables en kunnen we er dan naar verwijzen of kunnen we een tabelletje toevoegen met deze variablen?}
The simulation variables differ per simulation, and have no fixed initialisation settings.


\paragraph{Input}
\label{par:method:model:details:input}
% \algemeen{The dynamics of many IBMs are driven by some environmental conditions which change over space and time. All these environmental conditions are “input”, i.e. imposed dynamics of certain state variables.}
This simulation requires a graph, sources and sinks in that graph, the locations of vertices in that graph, the human-autonomous ratio and optionally how long the simulation should run. The graph represents the connections between the streets, since we are not interested in graph visualisation one also needs to indicate the location of each vertex in the world space. Vertices can be marked as a source or as a sink, or both, making that vertex a possible starting point or destination for drivers. The human-autonomous ratio indicates how many human drivers we have relative to the number of autonomous cars. It controls the ratio of human cars to autonomous cars that are generated at each source.